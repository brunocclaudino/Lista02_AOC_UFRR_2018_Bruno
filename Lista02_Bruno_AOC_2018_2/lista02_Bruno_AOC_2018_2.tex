\documentclass[12pt,a4paper]{article}
\usepackage[T1]{fontenc}
\usepackage[utf8]{inputenc}
\usepackage[top=2cm, bottom=2cm, left=2.5cm, right=2.5cm]{geometry}
\usepackage{amsmath,amssymb}
\usepackage{enumerate}
\usepackage[brazil]{babel}
\usepackage{graphicx}
\usepackage{indentfirst}

\begin{document}
\begin{titlepage}
\begin{center}

\textbf{Lista 02 - AOC 2018.2\\
Bruno Cesar da Silva Claudino - 1201424425\\}
\vspace{2cm}
\textbf{Respostas}
	
\end{center}
\begin{enumerate}[1)]
	\item
	\textbf{As vantagens de um processador multiciclo com a relação ao processador uniciclo ,são que diferentemente do uniciclo que nem todos os componentes são executados em uma execução de instrução,visto que a execução da instrução é feita em um único ciclo de clock . Na execução do multiciclo as instruçoes são divididas em 5 etapas: Busca de instrução, decodificação, busca de operandos, execução e armazenamento do resultado. Em um ciclo de clock é executado um componente para a estapa que esta sendo solicitada, logo cada ciclo faz menos coisas, isso quer dizer que os ciclos podem ser mais curtos. Então como o multiciclo é dividido em 5 etapas e cada etapa é realizada em um ciclo de clock, assim, o ciclo de clock do multicilo é 6 vezes mais curto que o do uniciclo, sua implementação é semelhante ao uniciclo, porem com uma quantidade menor de componentes no caminho de dados, mas sua implementação possui uma lógica de complexidade maior. O multiciclo ainda pode implementar pipeline que é uma técnica que permite a execução de instruções em paralelismo.}
	\item
	\textbf{Para a implementação do pipeline no multiciclo são necessárias as seguintes modificações no multiciclo:\\}
	\begin{itemize}
		\item \textbf{Registradores que funcionarão como buffers temporários, onde serão armazenados os resultados dos componentes e o estado em que a instrução se encontra.}
		\item \textbf{Flags para a UC(Unidade de Controle) ter o controle dos registradores adicionais.}
		\item \textbf{Tratamento de stalls evitando a parada do pipeline e das instruções, garantindo um funcionamento eficiente do pipeline.}
	\end{itemize}
	\item \textbf{Para um processador sem o pipeline funcionará da seguinte maneira:
	Serão um ciclo de clock para cada uma das etapas de uma instrução, então:}
	\begin{center}
	\begin{tabular}{|c|c|c|}
		\hline
		Linhas & Instrução & Ciclos \\ \hline
		1 & subi \$t2,\$t2,4 & 4 \\ \hline
		2 & lw \$t1,0(\$t2) & 5 \\ \hline
		3 & add \$t3,\$t1,\$t4 & 4 \\ \hline
		4 & add \$t4,\$t3,\$t3 & 4 \\ \hline
		5 & sw \$t4,0(\$t2) & 4 \\ \hline
		6 & beq \$t2,\$0,loop & 3 \\ \hline
		- & Total & 24 \\ \hline			
	\end{tabular}
	\end{center}
	\textbf{Para um processador com o pipeline funcionará da seguinte maneira:}
	\begin{center}
		\begin{tabular}{|c|c|c|}
			\hline
			Ciclos & Etapa(Linha) & Consequente \\ \hline
			1 & IF(1)&  \\ \hline
			2 & ID(1), IF(2)&  \\ \hline
			3 & EX(1), STALL(2) & Conflito de dados.Depende do resultado da linha 1\\ \hline
			4 & WB(1), STALL(2) &  \\ \hline
			5 & ID(2), IF(3) &  \\ \hline
			6 & EX(2), STALL(3) &  Conflito de dados.Depende do resultado da linha 2. \\ \hline
			7 & MEM(2), STALL(2) &  \\ \hline
			8 & WB(2), STALL(3) &  \\ \hline
			9 & ID(3), IF(4) &  \\ \hline
			10 & EX(3), STALL(4)&Conflito de dados.Depende do resultado da linha 3\\ \hline
			11 & WB(3), STALL(4) &  \\ \hline
			12 & ID(4), IF(5) &  \\ \hline
			13 & EX(4), ID(5) &  \\ \hline
			14 & WB(4), EX(5) &  \\ \hline
			15& MEM(5),STALL(6)&Conflito estrutural.2 instruções acessando a memória juntamente. \\ \hline
			16 &  IF(6)&  \\ \hline
			17 & ID(6) &  \\ \hline
			18 & EX(6) &  \\ \hline			
		\end{tabular}
	\end{center}
	\textbf{Logo o speedup é de  }$\frac{24}{18}\cong 1,33$.
	
	\item 
	\begin{itemize}
		\item
		\textbf{RAW:\\(sub.d e div.d): para a sub.d ser calculada é necessário que a div.d calcule o valor para F1.}
		\item
		\textbf{WAW:\\(s.d e sub.d): para s.d é necessário que o valor de F4 seja calculado na linha 2.}
		\item
		\textbf{WAR:
			\\(add.d e sub.d): O valor de F5 é sobrescrito no linha 4.
			\\Na linha 5 possui WAW onde F4 é sobrescrito na linha 3, RAW, onde F5 precisa ser calculado na linha 4.}
		\item
		\textbf{\\Na linha 5 possui WAW onde F4 é sobrescrito na linha 3, RAW, onde F5 precisa ser calculado na linha 4.}
	\end{itemize}
	 \item 
	 \textbf{\\$CPU execution time = (CPU clock cycles + Memory stall cycles*Clock cycle)
	 	\\CPU execution time =(IC*CPI)*Clock cycle
	 	\\CPU execution time =IC*1*Clock cycle
	 	\\
	 	\\Memory stall cycles = IC*(Memocy accesses/Instruction)*Miss rate * miss penalty
	 	\\Memory stall cycles = IC*(1+0.5)*0.02*25
	 	\\Memory stall cycles = IC*0.75
	 	\\
	 	\\CPU execution time(cache) = (IC*1.0 + IC*0.75)* Clock cycle
	 	\\CPU execution time(cache) = 1.75*IC* Clock cycle
	 	\\$}
 	\begin{center}	
 	\begin{tabular}{|c|}
 		\hline
 		\\$\frac{CPU execution time(cache)}{CPU execution time} = \frac{1.75*IC* Clock cycle}{1*IC* Clock cycle} = 1.75$
		\\\\ 		\hline
	\end{tabular} 	
	\end{center}	
	 
	 \item 
	 \begin{itemize}
	 	\item
	 	\textbf{Write through: escrita ao mesmo tempo no cache e na memória.}
	 	\begin{itemize}
	 		\item 
	 		\textbf{Vantagens:}
	 		\begin{itemize}
	 			\item 
	 			\textbf{Fácil de implementar.}
	 			\item 
	 			\textbf{Um "cache-miss" nunca resulta em escritas na memória.}
	 			\item 
	 			\textbf{A memória tem sempre a informação mais recente.}
	 		\end{itemize}
 		\item
 		\textbf{Desvantagens:}
 		\begin{itemize}
 			\item 
 			\textbf{A escrita é lenta;}
 			\item 
 			\textbf{Cada escrita necessita de um acesso à memória.}
 			\item 
 			\textbf{Consequentemente usa mais largura de banda da memória.}
 		\end{itemize}
 	 	\end{itemize}
  	
  	\item
  	\textbf{Write back: escrita somente no cache e atualização na memória somente na substituição do bloco.}
  	\begin{itemize}
  		\item 
  		\textbf{Vantagens:}
  		\begin{itemize}
  			\item 
  			\textbf{A escrita ocorre à velocidade do cache;}
  			\item 
  			\textbf{Escritas múltiplas de um endereço requerem apenas uma escrita na memória;}
  			\item 
  			\textbf{Consome menos largura de banda.}
  		\end{itemize}
  		\item
  		\textbf{Desvantagens:}
  		\begin{itemize}
  			\item 
  			\textbf{Difícil de implementar;}
  			\item 
  			\textbf{Nem sempre existe consistência entre os dados existentes no cache e na memória;}
  			\item 
  			\textbf{Leituras de blocos de endereços no cache podem resultar em escritas de blocos de endereços "dirty" na memória.}
  		\end{itemize}
  	\end{itemize}
 	\item
  	\textbf{Localidade Temporal: Se um item é referenciado, ele tende a ser referenciado novamente dentro de um espaço de tempo curto. Se o estudante tiver trazido o livro recentemente para sua mesa, é provável que o faça em breve novamente.}
  	\item
  	\textbf{Localidade Espacial: Se um item é referenciado, itens cujos endereços sejam próximos dele tendem a ser logo referenciados.}
	 	
	 \end{itemize}
\end{enumerate}
\end{titlepage}
\end{document}